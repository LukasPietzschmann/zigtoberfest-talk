\documentclass[aspectratio=169]{beamer}

\makeatletter
\appto\input@path{{pkgs/awesome-beamer},{pkgs/smile}}
\makeatother

\definecolor{dgreen}{HTML}{081c15}
\usetheme[english,color,coloraccent=dgreen,notoc,hyperref,listings]{awesome}

\usepackage[
	backend=biber,
	style=alphabetic
]{biblatex}
\usepackage[strict,autostyle]{csquotes}
\nocite{*}
\addbibresource{refs.bib}

\usetikzlibrary{tikzmark}
\def\m#1{\tikzmark{#1}}
\newcommand<>\ul[5][0pt]{\begin{tikzpicture}[o,blend mode=multiply]
	\coordinate(#5_west) at ([yshift=#1-0.25ex]pic cs:#3);
	\coordinate(#5_east) at ([yshift=#1+0.35ex]pic cs:#4);
	\fill[ul=#2,visible on=#6] (#5_west) rectangle node(#5){} (#5_east);
\end{tikzpicture}}
\tikzset{ul/.style={#1!70!black,opacity=0.40,rounded corners=.17ex}}

\usepackage{fontawesome5}
\usepackage{tikzpingus}

\usepackage[sfdefault]{FiraSans}

\usepackage[duration=25,defaulttransition=fade]{pdfpc}
\newcommand<>{\talknote}[1]{\only#2{\pdfpcnote{- #1}\relax}}

\background{uni.jpg}
\title{Constrained Zig}
\subtitle{Exploring properties of constraint programs}
\author{Lukas Pietzschmann}
\email{lukas.pietzschmann@uni-ulm.de}
\institute{Zigtoberfest}
\uni{Hochschule München}
\location{Munich}
\date{October 19$^\text{th}$, 2024}

\makeatletter
\newsavebox\avatar
\savebox\avatar{\begin{tikzpicture}
	\node[roundnode,fill=accent,inner sep=0,outer sep=0,minimum width=1.75cm+\smile@linewidth] at (0,0) {};
	\clip (0,0) circle (\dimexpr1.75cm/2\relax);
	\node at (0,0) {\includegraphics[width=1.75cm]{me.png}};
\end{tikzpicture}}

\def\link #1 to #2;{\def\ULdepth{.5pt}\def\ULthickness{.1pt}\uline{\href{#2}{#1}}}

\def\amcard#1#2#3{\begin{tikzpicture}
	\node (T) {\faCaretRight~#2};
	\node[below=1mm of T.south west,anchor=north west,text width=\textwidth/2-8.5mm,align=justify] (D) {\small#3};
	\node[roundednode,fit=(T)(D),node on layer=background,fill=lightgray!20] (F) {};
	\node[node on layer=background,anchor=north east,scale=2,opacity=0.1] at (F.north east) {#1};
\end{tikzpicture}}

\tikzset{bb/.style={draw=tcbcolframe,dash pattern=on 1mm off 1mm,dash phase=0.5mm,tcb@spec,segmentation@style}}
\def\setlinetext#1{\small\color{tcbcolframe}\contourlength{1.5pt}\contour{tcbcolback}{#1}}
\newenvironment{versusbox}[1][]{
	\begin{beamerbox}[sidebyside,segmentation code={
		\path[bb] (segmentation.north) to node[rotate=90]{\setlinetext{#1}} (segmentation.south);
	}]{gray}{}%
}{\end{beamerbox}}

\usepackage[verbatim]{lstfiracode}
\lstdefinestyle{firastyleb}{style=FiraCodeStyle,style=smile@lst@base}
\lstdefinestyle{firastylep}{style=FiraCodeStyle,style=smile@lst@plain}
\lstset{tabsize=4,style=firastylep}
\lstdefinelanguage{chr}{
	keywords={/, <=>, true}
}
\lstnewenvironment{chr}{\lstset{language=chr}}{}
\newcommand\chri[2][]{\lstinline[language=chr,#1]{#2}}

\def\signed#1{{\leavevmode\unskip\nobreak\hfil\penalty50\hskip2em
	\hbox{}\nobreak\hfil(#1)%
	\parfillskip=0pt \finalhyphendemerits=0 \endgraf}}

\newsavebox\quotebox
\newenvironment{aquote}[1]
	{\savebox\quotebox{#1}\begin{quote}}
	{\signed{\usebox\quotebox}\end{quote}}

\begin{document}

\makeatletter

\maketitle

\section{About Me}
\begin{frame}
	\frametitle{About Me}
	\begin{tikzpicture}[node distance=2mm]
		\node[roundednode,shadow,draw=none,fill=lightgray,node on layer=background,minimum width=\textwidth,minimum height=2cm] (B) {};
		\node[xshift=1.75cm/2+0.25cm] at (B.west) {\usebox\avatar};
		\node[] at (B) {\bfseries Hi, I'm Lukas!};
		\begin{scope}[scale=0.6]
			\pingu[left wing wave,heart=accent,shift={(7cm,2mm)}]
		\end{scope}

		\node[below=of B.south west,anchor=north west] (S) {\amcard{\faUniversity}{Studies}{I'm studying computer science for my master's degree at Ulm University. Currently, I'm working on my thesis.}};
		\node[below=of B.south east,anchor=north east] (I) {\amcard{\faLightbulb}{Interests}{I enjoy various things, but compilers, typesetting, and functional programming spark the most joy inside me.}};
	\end{tikzpicture}
\end{frame}

\section{Motivation}
\begin{frame}[fragile]
	\frametitle[\cite{freuder1997}]{Motivation}
	\begin{aquote}{Eugene C. Freuder}
		Constraint Programming represents one of the closest approaches computer science
		has yet made to the holy grail of programming: the user states the problem, the
		computer solves it.
	\end{aquote}\bigskip
	\lstset{xleftmargin=15mm}\begin{chr}
§\m{ps1}§prime(N)§\m{pe1}§ §\m{rem1s}§\§\m{rem1e}§ §\m{ps2}§prime(M)§\m{pe2}§ §\m{simpa1s}§<=>§\m{simpa1e}§ §\m{gs1}§M % N = 0§\m{ge1}§ | §\m{rs1}§true§\m{re1}§.
	\end{chr}
	\ul<1->{blue}{ps1}{pe1}{p1}%
	\ul<1->{gray}{rem1s}{rem1e}{rem1}%
	\ul<1->{blue}{ps2}{pe2}{p2}%
	\ul<1->{gray}{simpa1s}{simpa1e}{simpa1}%
	\ul<1->{green}{gs1}{ge1}{g1}%
	\ul<1->{red}{rs1}{re1}{r1}%
	\begin{tikzpicture}[o,execute at begin node={\small}]
		\node[shift={(-2.2cm,-7mm)}] at (p1) (P1) {First prime constraint};
		\node[shift={(-1.5cm, -13mm)}] at (rem1) (REM1) {Simpagation};
		\node[shift={(2.5cm,-13mm)}] at (p2) (P2) {Second prime constraint};
		\node[shift={(1cm,-7mm)}] at (g1) (G1) {Guard};
		\node[shift={(1cm,-7mm)}] at (r1) (R1) {Result};

		\draw[textarrow=blue] (P1) -| (p1);
		\draw[textarrow=gray] (REM1) -| (rem1);
		\draw[textarrow=blue] (P2) -| (p2);
		\draw[textarrow=green] (G1) -| (g1);
		\draw[textarrow=red] (R1) -| (r1);
	\end{tikzpicture}
\end{frame}

\section{CHR Syntax and Semantic}
\begin{frame}
	\frametitle[Rules]{Introduction to CHR}
	CHR knows about three kinds of rules:\par\medskip
	\begin{wide}
	\begin{columns}[c]
		\begin{column}{0.33\textwidth}
			\begin{block}[Propagation]
				\centerline{\chri{C => G | CN}}\medskip
				\chri{CN} is added if \chri{G} holds
			\end{block}
		\end{column}
		\begin{column}{0.33\textwidth}
			\begin{block}[Simplification]
				\centerline{\chri{C <=> G | CN}}\medskip
				\chri{CO} is replaced by \chri{CN} if \chri{G} holds
			\end{block}
		\end{column}
		\begin{column}{0.33\textwidth}
			\begin{block}[Simpagation]
				\centerline{\chri{C}$_1$ \chri{\\ C}$_2$ \chri{<=> G | CN}}\medskip
				\chri{CO} is replaced by \chri{CN} if \chri{G} holds
			\end{block}
		\end{column}
	\end{columns}
	\end{wide}\par\bigskip
	\begin{tabular}{r|l|l}
		C  & Head  & CHR Constraints                           \\
		G  & Guard & Conjunction of built-ins                  \\
		CN & Body  & CHR Constraints and conjunction built-ins
	\end{tabular}
	\begin{modal}<2>[Generalized Simpagation Rule]
		\begin{itemize}
			\item Propagation and simplification rules are just specialized simpagation rules
			\item Behind the curtaint, the embedding also only uses simpagations
			\item We, however, will always use the most specific rule
		\end{itemize}
	\end{modal}
	\onslide<2>\begin{tikzpicture}[o,scale=0.5]
		\pingu[small,heart=accent,halo=blue,staff right,right wing wave,shift={(2.5cm,-1.8cm)}] at (current page.south east)
	\end{tikzpicture}
\end{frame}

\begin{frame}
	\frametitle[Constraint Store]{Introduction to CHR}
\end{frame}

\section{References}
\begin{frame}[allowframebreaks]
	\frametitle{References}
	\printbibliography
\end{frame}
\end{document}
