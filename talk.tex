\documentclass[aspectratio=169]{beamer}

\makeatletter
\appto\input@path{{pkgs/awesome-beamer},{pkgs/smile}}
\makeatother

\definecolor{dgreen}{HTML}{081c15}
\usetheme[english,color,coloraccent=dgreen,notoc,hyperref,listings]{awesome}

\usepackage[
	backend=biber,
	style=alphabetic
]{biblatex}
\usepackage[strict,autostyle]{csquotes}
\nocite{*}
\addbibresource{refs.bib}

\usetikzlibrary{tikzmark,decorations.text}
\def\m#1{\tikzmark{#1}}
\newcommand<>\ul[5][0pt]{\begin{tikzpicture}[o,blend mode=multiply]
	\coordinate(#5_west) at ([yshift=#1-0.25ex]pic cs:#3);
	\coordinate(#5_east) at ([yshift=#1+0.35ex]pic cs:#4);
	\fill[ul=#2,visible on=#6] (#5_west) rectangle node(#5){} (#5_east);
\end{tikzpicture}}
\tikzset{ul/.style={#1!70!black,opacity=0.40,rounded corners=.17ex}}
\def\um#1#2#3{\begin{tikzpicture}[o]
	\coordinate(#3_west) at ([yshift=0ex]pic cs:#1);
	\coordinate(#3_east) at ([yshift=0ex]pic cs:#2);
	\coordinate(#3) at ($(#3_west)!0.5!(#3_east)$);
\end{tikzpicture}}

\usepackage{emoji}
\usepackage{fontawesome5}
\usepackage{tikzpingus}
\usepackage{fancyqr}

\def\checkbox{\raisebox{-2pt}{\begin{tikzpicture}[]
	\node (B) {\color{accent}\scalebox{.65}{\faCheck}};
	\node[roundednode,node on layer=background,fill=lightgray!30,rnd,minimum width=1.7ex,minimum height=1.7ex] at (B) {};
\end{tikzpicture}}}

\usepackage[sfdefault]{FiraSans}
\usepackage{firamath-otf}
\usepackage{fontspec}
\setmonofont[
	Path = ./fonts/,
	Scale = .9,
	Extension = .ttf,
	Contextuals=Alternate,
	BoldFont={*-Bold},
	UprightFont={*-Regular},
]{Fira Code}

\usepackage[duration=25,defaulttransition=fade]{pdfpc}
\newcommand<>{\talknote}[1]{\only#2{\pdfpcnote{- #1}\relax}}

\background{uni.jpg}
\title{Constrained Zig}
\subtitle{Exploring properties of constraint programs}
\author{Lukas Pietzschmann}
\email{lukas.pietzschmann@uni-ulm.de}
\institute{Zigtoberfest}
\uni{Hochschule München}
\location{Munich}
\date{October 19$^\text{th}$, 2024}

\makeatletter
\newsavebox\avatar
\savebox\avatar{\begin{tikzpicture}
	\node[roundnode,fill=accent,inner sep=0,outer sep=0,minimum width=1.75cm+\smile@linewidth] at (0,0) {};
	\clip (0,0) circle (\dimexpr1.75cm/2\relax);
	\node at (0,0) {\includegraphics[width=1.75cm]{me.png}};
\end{tikzpicture}}

\def\link #1 to #2;{\def\ULdepth{.5pt}\def\ULthickness{.1pt}\uline{\href{#2}{#1}}}
\def\ergo{\raisebox{-1pt}{\faAngleRight}}
\newcommand<>\tc[2]{\textcolor#3{#1}{#2}}
\newcommand<>\cg[1]{\tc#2{gray}{#1}}
\newcommand<>\cb[1]{\tc#2{black}{#1}}

\def\amcard#1#2#3{\begin{tikzpicture}
	\node (T) {\faCaretRight~#2};
	\node[below=1mm of T.south west,anchor=north west,text width=\textwidth/2-8.5mm,align=justify] (D) {\small#3};
	\node[roundednode,fit=(T)(D),node on layer=background,fill=lightgray!20] (F) {};
	\node[node on layer=background,anchor=north east,scale=2,opacity=0.1] at (F.north east) {#1};
\end{tikzpicture}}

\tikzset{bb/.style={draw=tcbcolframe,dash pattern=on 1mm off 1mm,dash phase=0.5mm,tcb@spec,segmentation@style}}
\def\setlinetext#1{\small\color{tcbcolframe}\contourlength{1.5pt}\contour{tcbcolback}{#1}}
\newenvironment{versusbox}[1][]{
	\begin{beamerbox}[sidebyside,segmentation code={
		\path[bb] (segmentation.north) to node[rotate=90]{\setlinetext{#1}} (segmentation.south);
	}]{gray}{}%
}{\end{beamerbox}}

\usepackage[verbatim]{lstfiracode}
\lstdefinestyle{firastyleb}{style=FiraCodeStyle,style=smile@lst@base}
\lstdefinestyle{firastylep}{style=FiraCodeStyle,style=smile@lst@plain}
\lstset{tabsize=4,style=firastylep}
\lstdefinelanguage{chr}{
	keywords={/, <=>, true}
}
\lstnewenvironment{chr}{\lstset{language=chr}}{}
\newcommand\chri[2][]{\lstinline[language=chr,#1]{#2}}

\def\signed#1{{\leavevmode\unskip\nobreak\hfil\penalty50\hskip2em
	\hbox{}\nobreak\hfil(#1)%
	\parfillskip=0pt \finalhyphendemerits=0 \endgraf}}

\newsavebox\quotebox
\newenvironment{aquote}[1]
	{\savebox\quotebox{#1}\begin{quote}}
	{\signed{\usebox\quotebox}\end{quote}}

\begin{document}

\makeatletter

\maketitle

\section{About Me}
\begin{frame}
	\frametitle{About Me}
	\begin{tikzpicture}[node distance=2mm]
		\node[roundednode,shadow,draw=none,fill=lightgray,node on layer=background,minimum width=\textwidth,minimum height=2cm] (B) {};
		\node[xshift=1.75cm/2+0.25cm] at (B.west) {\usebox\avatar};
		\node[] at (B) {\bfseries Hi, I'm Lukas!};
		\begin{scope}[scale=0.6]
			\pingu[left wing wave,heart=accent,shift={(7cm,2mm)}]
		\end{scope}

		\uncover<2->{\node[below=of B.south west,anchor=north west] (S) {\amcard{\faUniversity}{Studies}{I'm studying computer science for my master's degree at Ulm University. Currently, I'm working on my thesis.}};}
		\uncover<3->{\node[below=of B.south east,anchor=north east] (I) {\amcard{\faLightbulb}{Interests}{I enjoy various things, but compilers, typesetting, and functional programming spark the most joy inside me.}};}
	\end{tikzpicture}
\end{frame}

\section{Motivation}
\begin{frame}[fragile]
	\frametitle[A first small example]{Motivation}
	\begin{onlyenv}<2->
		\begin{aquote}{Eugene C. Freuder~\cite{freuder1997}}
			Constraint Programming represents one of the closest approaches computer science
			has yet made to the holy grail of programming: the user states the problem, the
			computer solves it.
		\end{aquote}\bigskip
	\end{onlyenv}\onslide<3->
	\lstset{xleftmargin=15mm}\begin{chr}
§\m{ps1}§min(N)§\m{pe1}§ §\m{rem1s}§\§\m{rem1e}§ §\m{ps2}§min(M)§\m{pe2}§ §\m{simpa1s}§<=>§\m{simpa1e}§ §\m{gs1}§N <= M§\m{ge1}§ | §\m{rs1}§true§\m{re1}§
	\end{chr}
	\ul<4->{blue}{ps1}{pe1}{p1}%
	\ul<5->{gray}{rem1s}{rem1e}{rem1}%
	\ul<4->{blue}{ps2}{pe2}{p2}%
	\ul<5->{gray}{simpa1s}{simpa1e}{simpa1}%
	\ul<6->{green}{gs1}{ge1}{g1}%
	\ul<7->{red}{rs1}{re1}{r1}%
	\begin{tikzpicture}[o,execute at begin node={\small}]
		\node[shift={(-4mm,-7mm)},left,visible on=<4->] at (p1) (P1) {First min constraint};
		\node[shift={(-4mm, -13mm)},left,visible on=<5->] at (rem1) (REM1) {Operation};
		\node[shift={(4mm,-13mm)},right,visible on=<4->] at (p2) (P2) {Second min constraint};
		\node[shift={(4mm,-7mm)},right,visible on=<6->] at (g1) (G1) {Guard};
		\node[shift={(4mm,-7mm)},right,visible on=<7->] at (r1) (R1) {Result};

		\draw[textarrow=blue,visible on=<4->] (P1) -| (p1);
		\draw[textarrow=gray,visible on=<5->] (REM1) -| (rem1);
		\draw[textarrow=blue,visible on=<4->] (P2) -| (p2);
		\draw[textarrow=green,visible on=<6->] (G1) -| (g1);
		\draw[textarrow=red,visible on=<7->] (R1) -| (r1);
	\end{tikzpicture}
\end{frame}

\section{CHR Syntax and Semantic}
\begin{frame}[label={rules}]
	\frametitle[Rules]{Introduction to CHR}
	\onslide<2->CHR knows about three kinds of rules:\par\medskip
	\begin{wide}
	\begin{columns}[c]
		\begin{column}<3->{0.33\textwidth}
			\begin{block}[Propagation]
				\centerline{\chri{C => G | CN}}\medskip
				\chri{CN} is inferred if \chri{G} holds
			\end{block}
		\end{column}
		\begin{column}<4->{0.33\textwidth}
			\begin{block}[Simplification]
				\centerline{\chri{C <=> G | CN}}\medskip
				\chri{CO} is simplified to \chri{CN} if \chri{G} holds
			\end{block}
		\end{column}
		\begin{column}<5->{0.33\textwidth}
			\begin{block}[Simpagation]
				\centerline{\chri{C}$_1$ \chri{\\ C}$_2$ \chri{<=> G | CN}}\medskip
				\chri{C}$_2$ is simplified to \chri{CN} if \chri{G} holds
			\end{block}
		\end{column}
	\end{columns}
	\end{wide}\par\bigskip
	\begin{tabular}{r|l|l}
		\onslide<8->C  & Head  & CHR Constraints                           \\
		\onslide<9->G  & Guard & Conjunction of \m{bu1s}built-ins\m{bu1e}  \\
		\onslide<11->CN & Body  & CHR Constraints and conjunction of built-ins
	\end{tabular}\onslide<1->
	\ul<10->{gray}{bu1s}{bu1e}{bi}\begin{tikzpicture}[o]
		\node[shift={(23mm,5mm)},visible on=<10->] at (bi) (BI) {\small\color{darkgray}Zig code};
		\draw[textarrow,shorten <=0,visible on=<10->] (BI.west) to (bi_east);
	\end{tikzpicture}
	\begin{modal}<beamer:6|handout:2>[Generalized Simpagation Rule]
		\hskip5mm\par Behind the curtain, the embedding represents every rule as a 4-tuple:
		\[\langle KH, RH, G, B \rangle\]
		$KH$: Kept Head\hfill$RH$: Removed Head\hfill$G$: Guard\hfill$B$: Body\par\medskip
		A propagation rule would then look like this: $\langle KH, \emptyset, G, B \rangle$
	\end{modal}
	\onslide<beamer:6|handout:2>\begin{tikzpicture}[o,scale=0.5]
		\pingu[small,heart=accent,halo=blue,staff right,right wing wave,shift={(2.5cm,-1.8cm)}] at (current page.south east)
	\end{tikzpicture}
\end{frame}

\begin{frame}[fragile,t]
	\frametitle[Constraint Store]{Introduction to CHR}
	\vspace{7mm}
	\begin{columns}[T]
		\begin{column}{0.65\textwidth}
			\begin{tikzpicture}\tikzset{query/.style={roundednode,inner sep=1.5mm,fill=lightgray!25,draw=gray,minimum width=6.7cm,text width=6.7cm}}
				\def\q{\only<all:1>{\color{gray}~Query~\ldots}}
				\only<-4>{\node[query] (Q) {\strut\cg{\ergo}\q\ttfamily~\only<all:2->{min(3)}\only<all:3->{, min(1)}\only<all:4->{, min(5)}};}
				\only<all:5->{\node[query] (Q) {\strut\cg{\ergo}\color{gray}\ttfamily~\cb<all:5-7>{min(3)}, \cb<all:9-12>{min(1)}, \cb<all:14-16>{min(5)}};}
			\end{tikzpicture}
			{\ttfamily%
				\m{ps3}\cg<all:6-8,10-13,15-16>{min(N)}\m{pe3}
				\chri{\\}
				\m{ps4}\cg<all:6-8,10-13,15-16>{min(M)}\m{pe4}
				\chri{<=>}
				\m{gs2}\cg<all:6-8,10-13,15-16>{N}\m{ge2}
				\chri{<=}
				\m{rs2}\cg<all:6-8,10-13,15-16>{M}\m{re2}
				\chri|
				\chri{true}%
			}\um{ps3}{pe3}{x}\um{ps4}{pe4}{y}\um{gs2}{ge2}{z}\um{rs2}{re2}{a}\par\vskip7mm
			\ergo~\only<all:-4>{Add constraints to solve for}%
			\only<all:5-7,10-11,15-16>{Check if we can find a matching for the currently active constraint}%
			\only<all:8,12-13>{Fire the rule that matched}%
			\only<all:9,14,17>{Make next query constraint active}%
			\only<all:18>{The remaining constraints in the store are the solution}%
		\end{column}
		\begin{column}{0.25\textwidth}
			\begin{tikzpicture}
				\node[roundednode,lcr,dashed,fill=lightgray!25,draw=gray,minimum width=\awesome@sidebarwidth-3mm,minimum height=3cm,anchor=north west,xshift=\awesome@textmargin] at ([yshift=-1.5cm]current page.north west) (S) {};
				\node[roundednode,color=black,text=white] at (S.south) {\tiny Constraint Store};

				\node[anchor=north,yshift=-2mm,visible on=<all:8-11>] at (S.north) {\ttfamily min(3)};
				\node[anchor=north,yshift=-2mm,visible on=<all:13->] at (S.north) {\ttfamily min(1)};
			\end{tikzpicture}
		\end{column}
	\end{columns}
	\begin{tikzpicture}[o]
		\node[visible on=<all:6>] at (x) {\bfseries\chri{min(3)}};
		\node[visible on=<all:10-12>] at (x) {\bfseries\chri{min(1)}};
		\node[visible on=<all:16>] at (x) {\bfseries\chri{min(1)}};

		\node[visible on=<all:7>] at (y) {\bfseries\chri{min(3)}};
		\node[visible on=<all:11-12>] at (y) {\bfseries\chri{min(3)}};
		\node[visible on=<all:15-16>] at (y) {\bfseries\chri{min(5)}};

		\node[visible on=<all:6>] at (z) {\bfseries\chri{3}};
		\node[visible on=<all:10-12>] at (z) {\bfseries\chri{1}};
		\node[visible on=<all:16-16>] at (z) {\bfseries\chri{1}};

		\node[visible on=<all:7>] at (a) {\bfseries\chri{3}};
		\node[visible on=<all:11-12>] at (a) {\bfseries\chri{3}};
		\node[visible on=<all:15-16>] at (a) {\bfseries\chri{5}};
	\end{tikzpicture}
\end{frame}

\section{The Embedding}
\tikzset{paper/.style={roundednode,anchor=south east,rotate=#1,shadow}}
\newsavebox\paper
\savebox\paper{\begin{tikzpicture}
	\node[paper=-10,xshift=4mm] (3) {\includegraphics[page=3,width=2.5cm]{FreeCHR.pdf}};
	\node[paper=-05,xshift=2mm] (2) {\includegraphics[page=2,width=2.5cm]{FreeCHR.pdf}};
	\node[paper=-00,xshift=0mm] (1) {\includegraphics[page=1,width=2.5cm]{FreeCHR.pdf}};
\end{tikzpicture}}
\begin{frame}
	\frametitle{Inner workings of the Embedding}
	\begin{wide}
	\begin{columns}[c]
		\begin{column}{\dimexpr\textwidth-\wd\paper\relax}
			\begin{itemize}[<+(1)->]
				\item Based on the concepts established in~\citetitle{rechenberger2023}~\cite{rechenberger2023}
			\end{itemize}\par\bigskip
			\begin{itemize}[<+(1)->]
				\item Every rule is represented as a 4-tuple (see slide \ref{rules<2>})
				\item We then compose more complex programs from subprograms --- or ultimately from single rules
				\item Lastly, we apply the composition to a state until a fixpoint is reached
			\end{itemize}
		\end{column}
		\begin{column}<2->{\wd\paper}
			\usebox\paper
		\end{column}
	\end{columns}
	\end{wide}
\end{frame}

\newsavebox\pingub
\savebox\pingub{\begin{tikzpicture}
	\pingu[heart=accent,sun glasses,head band=red,wings wave,banner=Check it out!]
\end{tikzpicture}}
\begin{frame}
	\frametitle{ZigCHR online}
	\begin{wide}
	\begin{columns}[c]
		\begin{column}{\dimexpr\textwidth-4cm\relax}
			\begin{itemize}
				\item The embedding's source code is available on \link GitHub to https://github.com/LukasPietzschmann/zigchr;
				\item Clone it and try it out yourself!
			\end{itemize}\par\medskip
			\begin{itemize}
				\item Also, the repo contains some of the examples we're discussing today
			\end{itemize}
		\end{column}
		\begin{column}{4cm}
			\begin{center}
				\fancyqr[height=3cm,color=accent]{https://lukas.pietzschmann.org/talks/zigtoberfest}\par
				\parbox{4cm}{\centering\tiny\url{https://lukas.pietzschmann.org/talks/zigtoberfest}}
			\end{center}
		\end{column}
	\end{columns}
	\end{wide}
	\begin{tikzpicture}[o]
		\node[shift={(6mm,1.3cm)},rotate=-30,scale=0.5] at (current page.south west) (P) {\usebox\pingub};
	\end{tikzpicture}
\end{frame}

\section{Some interesting examples}
\begin{frame}
	\frametitle{Anytime Algorithms}
	\begin{itemize}[<+(1)->]
		\item We can interrupt the execution at anytime
		\item The intermediate state will be an approximation of the final result
		\item After that, the program will continue like nothing happened
	\end{itemize}
	\begin{itemize}
		\item<5-> This is especially useful if we need to guarantee a certain response time
	\end{itemize}
\end{frame}

\begin{frame}[fragile]
	\frametitle{Easy Memorization}
	\begin{itemize}
		\item[\ergo]<3-> Quizz: Change three symbols to make this have linear complexity
	\end{itemize}\par\bigskip
	\begin{onlyenv}<{2,3}>
	\begin{chr}
		fib(0, M) <=> M = 1
		fib(1, M) <=> M = 1
		fib(N, M) <=> N >= 2 |
			fib(N-1,M1), fib(N-2,M2), M is M1 + M2
	\end{chr}
	\end{onlyenv}
	\begin{onlyenv}<4->
	\begin{chr}
		fib(0, M) ==> M = 1
		fib(1, M) ==> M = 1
		fib(N, M) ==> N >= 2 |
			fib(N-1,M1), fib(N-2,M2), M is M1 + M2
	\end{chr}
	\end{onlyenv}
	\par\bigskip
	\begin{itemize}
		\item<4-> We have to change the simplification rules to propagations
		\item<5-> This way, we do not remove calculated values from the store
		\item<6-> But we remember them for future use
	\end{itemize}
\end{frame}

\begin{frame}[fragile]
	\frametitle[What Confluence is]{Confluence}
	\begin{wide}
	\begin{columns}[c]
		\begin{column}<2->{0.48\textwidth}
			\begin{block}[Rule order matters]
			\begin{chr}
				throw(Coin) <=> head
				throw(Coin) <=> tail
			\end{chr}
			\begin{itemize}[<+(2)->]
				\item Depending on which rule we check first, we get a different result
				\item That's great for coin tossing
				\item Not so much for determinism
			\end{itemize}
			\end{block}
		\end{column}
		\begin{column}<+(2)->{0.48\textwidth}
			\begin{block}[Constraint order matters]
			\begin{chr}
				set(K,V), c(K,V') <=> c(K,V)
			\end{chr}
			\begin{itemize}[<+(2)->]
				\item Imagine the following query:\newline\chri{c(x,0),set(x,1),set(x,2)}
				\item Depending on what constraint we select first, \chri{x} will be set differently
			\end{itemize}
			\end{block}
		\end{column}
	\end{columns}\par\bigskip
	\onslide<+(2)->{\ergo~In confluent programs, the relation between the initial and final state is a function}
	\end{wide}
\end{frame}

\begin{frame}[fragile]
	\frametitle[Achieve Confluence]{Confluence}
	\begin{columns}[t]
		\begin{column}{0.7\textwidth}
			\begin{itemize}
				\item<3-> Both rules are applicable to a state containing only \chri{a}
				\item<4-> From this, we can derive the two new states
				\item<5-> Confluence requires then to be joinable
			\end{itemize}
		\end{column}
		\begin{column}<2->{0.2\textwidth}
			\begin{chr}
				a => b
				a => false
			\end{chr}
		\end{column}
	\end{columns}\par\vskip6mm\onslide<6->{\centerline{\textcolor{gray}{\rule{\textwidth-1cm}{\smile@linewidth}}}}
	\begin{columns}[t]
		\begin{column}{0.7\textwidth}
			\begin{itemize}
				\item<6-> We can make the two states joinable by adding a single rule
			\end{itemize}
		\end{column}
		\begin{column}<7->{0.2\textwidth}
			\begin{chr}
				a => b
				a => false
				b => false
			\end{chr}
		\end{column}
	\end{columns}
	\begin{tikzpicture}[o]
		\node[shift={(\awesome@sidebarwidth/2+6mm,16mm)},visible on=<3->] at (current page.west) (a) {\chri{a}};
		\node[yshift=-2cm,visible on=<5->] at (a) (e) {\chri{false} $\not\equiv$ \chri{b}};
		\coordinate (m) at ($ (a)!0.5!(e) $);
		\node[xshift=7mm,visible on=<4->] at (m) (x) {\chri{b}};
		\node[xshift=-7mm,visible on=<4->] at (m) (y) {\chri{false}};

		\node[roundednode,draw=gray,fill=lightgray!20,node on layer=background,dashed,lcr,fit=(a)(e)(x)(y),visible on=<3->] {};

		\draw[arrow,visible on=<4->] (a) -- (x);
		\draw[arrow,visible on=<4->] (a) -- (y);
		\draw[arrow,visible on=<5->] (x) -- (e);
		\draw[arrow,visible on=<5->] (y) -- (e);
	\end{tikzpicture}
\end{frame}

\begin{frame}[fragile]
	\frametitle[Gotchas]{Confluence}
	\begin{itemize}\setlength\itemsep{1em}
		\item<2->[\checkbox] Not every program can be made confluent
			\begin{visibleenv}<3->
			\begin{chr}
				a => true
				a => false
			\end{chr}
			\end{visibleenv}
		\item<4->[\checkbox] Be aware of the semantics of your program
			\begin{visibleenv}<5->
			\begin{chr}
				set(K, V), c(K, V') <=> c(K, V)
			\end{chr}
			\faCaretUp~does not equal~\faCaretDown
			\begin{chr}
				set(K, V), c(K, V') <=> V = V' | c(K, V)
			\end{chr}
			\end{visibleenv}
	\end{itemize}
\end{frame}

\begin{frame}
	\frametitle[Why we need it]{Confluence}
	\begin{wide}
	\begin{itemize}
		\item<2-> If a program terminates, we can check if it's confluent
		\item<3-> If it is, we get more cool properties for free:
	\end{itemize}
	\begin{columns}[b]
		\begin{column}<4->{0.49\textwidth}
			\begin{block}[Incrementality]
				\begin{itemize}
					\item We can add constraints during the program's execution
					\item<5-> They will eventually participate in the computation
					\item<6-> The result will be the same as if was there from the beginning
				\end{itemize}
			\end{block}
		\end{column}
		\begin{column}<7->{0.49\textwidth}
			\begin{tikzpicture}[o,scale=.6]
				\pingu[wings wave,princess crown,eyes shiny,xshift=15mm]
			\end{tikzpicture}
			\begin{block}[Parallelism]
				\begin{itemize}
					\item The algorithm can be executed in parallel
					\item<8-> Without any modifications
				\end{itemize}
			\end{block}
		\end{column}
	\end{columns}
	\end{wide}
\end{frame}

\newsavebox\mapredb
\savebox\mapredb{\begin{tikzpicture}[node distance=5mm and 1cm]
	\tikzset{
		sarrow/.style={arrow,short=2pt},
		map/.style={draw=blue!13!black,fill=blue!2},
		reduce/.style={draw=red!13!black,fill=red!2},
	}
	\node[roundednode,dashed,lcr,fill=lightgray!20,draw=gray,inner sep=2mm] (F) {\Large\faFile*[regular]};
	\node[roundednode,right=of F] (SP) {\rotatebox{-90}{Split}};
	\node[roundednode,map,right=of SP] (M3) {Map};
	\node[roundednode,map,above=of M3] (M2) {Map};
	\node[roundednode,map,above=of M2] (M1) {Map};
	\node[roundednode,map,below=of M3] (M4) {Map};
	\node[roundednode,map,below=of M4] (M5) {Map};
	\node[roundednode,right=of M3] (SH) {\rotatebox{-90}{Shuffle}};
	\node[roundednode,reduce,right=of SH] (R2) {Reduce};
	\node[roundednode,reduce,above=of R2] (R1) {Reduce};
	\node[roundednode,reduce,below=of R2] (R3) {Reduce};
	\node[roundednode,right=of R2,dashed,lcr,fill=lightgray!20,draw=gray,inner sep=2mm] (E) {\Large\faFile*[regular]};

	\draw[sarrow] (F) -- (SP);
	\foreach \x in {1,2,3,4,5} {
		\draw[sarrow] (SP) -- (M\x);
		\draw[sarrow] (M\x) -- (SH);
	}
	\foreach \x in {1,2,3} {
		\draw[sarrow] (SH) -- (R\x);
		\draw[sarrow] (R\x) -- (E);
	}
\end{tikzpicture}}
\begin{frame}[fragile]
	\frametitle[Map-Reduce in CHR]{Declarative Concurrency}
	\begin{visibleenv}<6->
	\begin{tikzpicture}\tikzset{query/.style={roundednode,inner sep=1.5mm,fill=lightgray!25,draw=gray,minimum width=\textwidth-2mm}}
		\node[query] (Q) {\strut};
		\node[anchor=west] at (Q.west){\strut\cg{\ergo}~\chri{map(square), reduce(add), v(2), v(3), v(4), v(5)}};
	\end{tikzpicture}
	\end{visibleenv}
	\begin{columns}[c]
		\begin{column}{0.68\textwidth}
			\begin{visibleenv}<4->
			\begin{chr}
				map(OP) \ v(X) <=>
					C =.. [OP, X, R],
					call(C),
					r(R)
			\end{chr}
			\end{visibleenv}
			\begin{visibleenv}<5->
			\begin{chr}
				reduce(OP) \ r(X), r(Y) <=>
					C =.. [OP, X, Y, R],
					call(C),
					r(R)
			\end{chr}
			\end{visibleenv}
		\end{column}
		\begin{column}<7->{0.25\textwidth}
			\begin{tikzpicture}
				\node[roundednode,lcr,dashed,fill=lightgray!25,draw=gray,minimum width=\awesome@sidebarwidth+4mm,minimum height=3cm,anchor=north west,xshift=\awesome@textmargin] at ([yshift=-1.5cm]current page.north west) (S) {};
				\node[roundednode,color=black,text=white] at (S.south) {\tiny Constraint Store};

				\node[anchor=north,yshift=-2mm] at (S.north) {\chri{reduce(add)}};
				\node[anchor=north,yshift=-7mm] at (S.north) {\chri{map(square)}};
				\node[anchor=north,yshift=-12mm] at (S.north) {\chri{r(54)}};
			\end{tikzpicture}
		\end{column}
	\end{columns}
	\mode<beamer>{\begin{modal}<2>
		\centerline{\usebox\mapredb}
	\end{modal}}
\end{frame}

\section{TL;DR}
\begin{frame}
	% \frametitle{Why should anyone use this?}
	\begin{wide}
	\centerline{\Huge\bfseries Why should anyone use this?}
	\end{wide}
\end{frame}

\newsavebox\epingub
\savebox\epingub{\begin{tikzpicture}
	\pingu[heart=accent,wings wave]
\end{tikzpicture}}
\begin{frame}
	\frametitle{What's left to say}
	\begin{wide}
		\begin{tikzpicture}[o]
			\path[postaction={decorate},decoration={text along path, text={|\bfseries\Huge\color{accent}|Zig is awesome},text align={fit to path}},visible on=<2->]
				([shift={(-2.7cm,.2mm)}]current page.center) to[bend left=55] ([shift={(2.7cm,.2mm)}]current page.center);
			\node[yshift=-7mm,visible on=<2->] (P) at (current page.center) {\usebox\epingub};
			\node[below=of P,visible on=<3->] {\bfseries\Large\color{accent}Constraint programming, too \emoji{wink}};
		\end{tikzpicture}
	\end{wide}
\end{frame}

\section{References}
\defbibheading{bibliography}[\bibname]{}

\begin{frame}[allowframebreaks]
	\frametitle{References}
	\printbibliography
\end{frame}
\end{document}
